\documentclass{anstrans}
\usepackage{graphicx}
\usepackage{booktabs}
\usepackage{microtype}
\usepackage{upgreek}
\usepackage[capitalise,nameinlink]{cleveref}

\newcommand{\efr}{E_\text{fr}}
\newcommand{\enp}{E_\text{n,p}}
\newcommand{\ened}{E_\text{n,d}}
\newcommand{\egp}{E_\text{$\gamma$,p}}
\newcommand{\egd}{E_\text{$\gamma$,p}}
\newcommand{\eb}{E_\upbeta}
\newcommand{\enu}{E_\nu}

%%%%%%%%%%%%%%%%%%%%%%%%%%%%%%%%%%%%%%%%%%%%%%%%%%%%%%%%%%%%%%%%%%%%%%%%%%%%%%%%
\title{Energy Deposition in the OpenMC Monte Carlo Particle Transport Code}
\author{Paul K.~Romano,$\!^{*}$ Andrew E. Johnson Amanda L.~Lund,$\!^{*}$ Someone Else,$^{\dagger}$ and Someone Else2,$^{\dagger}$}

\institute{
$^{*}$Computational Science Division, Argonne National Laboratory, 9700 S Cass Ave., Lemont, IL 60439, promano@anl.gov
\and
$^{\dagger}$MIT, ...
}

% Optional disclaimer: remove this command to hide
\disclaimer{Notice: this manuscript is a work of fiction. Any resemblance to
actual articles, living or dead, is purely coincidental.}


\begin{document}
%%%%%%%%%%%%%%%%%%%%%%%%%%%%%%%%%%%%%%%%%%%%%%%%%%%%%%%%%%%%%%%%%%%%%%%%%%%%%%%%
\section{Introduction}

OpenMC~\cite{romano2015ane1}. Photon transport~\cite{lund2018anl}. Data
processing capabilities~\cite{romano2017epjwoc}.

As particles traverse a problem, some portion of their energy is deposited at
collision sites. This energy is deposited when charged particles, including
electrons and recoil nuclei, undergo electromagnetic interactions with
surrounding electons and ions. The amount of energy that is deposited for a
specific reaction is referred to as the ``heating number'' and can be computed
using a program like NJOY with the HEATR module. The total heating rate can be
written as,
\begin{equation}
    H(E) = \phi(E)\sum_i \rho_i \sum_r k_{i, r}(E),
\end{equation}
where $\phi(E)$ is the scalar flux at energy $E$, $\rho_i$ is the density of
nuclide $i$, and $k_{i, r}$ is the KERMA (Kinetic Energy Release in Materials)
coefficient (cite Mack97) for reaction $r$ of isotope $i$. The heating rate,
$H(E)$, has units of energy per time, typically eV/s. KERMA has units of energy
$\times$ cross-section (e.g., eV-barn) and can therefore be used much like a
reaction cross section for the purpose of tallying energy deposition.

KERMA coefficients can be computed using a nuclear data processing code like
NJOY, which uses an energy-balance method to calculates KERMA as
\begin{equation}
    k_r(E) = \left(E + Q_r - \bar{E}_{r, n}
    - \bar{E}_{r, \gamma}\right)\sigma_{r}(E),
\end{equation}
where $Q_r$ is the $Q$ value of reaction $r$, $\bar{E}_{r,n}$ is the average
energy carried away by secondary neutrons, $\bar{E}_{r,\gamma}$ is the average
energy carried away by secondary photons, and $\sigma_r(E)$ is the microscopic
cross section for reaction $r$. Without loss of generality, we have omitted the
$i$ subscript for simplicity, and it is to be understood that $k_r$ represents
the KERMA for a specific nuclide. The term in parentheses is known as the
\emph{heating number} and represents the energy put in to the reaction (the
incident neutron energy and the $Q$ value) and the energy transported away by
secondary neutral particles.

During a fission event, there are potentially many secondary particles, and all
must be considered. The total energy released in a fission event is typically
broken up into the following categories:
\begin{equation*}
\begin{split}
  \efr  &\equiv \text{energy of fission fragments} \\
  \enp  &\equiv \text{energy of prompt fission neutrons} \\
  \ened &\equiv \text{energy of delayed fission neutrons} \\
  \egp  &\equiv \text{energy of prompt fission photons} \\
  \egd  &\equiv \text{energy of delayed fission photons} \\
  \eb   &\equiv \text{energy of released $\upbeta$ particles} \\
  \enu  &\equiv \text{energy of neutrinos}
\end{split}
\end{equation*}
These components are defined in MF=1, MT=458 data in an ENDF evaluation. All
these quantities may depend upon incident neutron energy, but this dependence is
not shown to make the following demonstrations cleaner. NJOY assumes that the
$Q$ value for fission is equal to the prompt energy release minus the incident
neutron energy:
\begin{equation}
    Q_f = \efr + \enp + \egp - E
\end{equation}
which results in the following expression for the fission KERMA:
\begin{equation}
    k_f(E) = \left[\efr + \enp + \egp - \bar{E}_{f,n} - \bar{E}_{f,\gamma}\right]\sigma_f(E).
\end{equation}
If the secondary neutron and photon yields and energies are consistent with the
components of fission energy release in MT=458, we then have $\enp = \bar{E}_n$
and $\egp = \bar{E}_p$, which results in a fission KERMA of
\begin{equation}
    \label{eq:njoy-kerma}
    k_f(E) = \efr \sigma_f(E).
\end{equation}

There are several problems with the use of \cref{eq:njoy-kerma} to tally energy
deposition in a Monte Carlo transport simulation. While \cref{eq:njoy-kerma}
would be appropriate for a coupled neutron-photon calculation wherein the energy
from photons is deposited separately, if we want to estimate the total heating
in a neutron-only calculation \cref{eq:njoy-kerma} ignores the energy of
photons. Additionally, even in a coupled neutron-photon calculation, the yield
and energy spectrum of delayed photons from fission is generally not known, so
tallying the energy from photons produced in neutron reactions will
underestimate the total heating. In this paper, we discuss how these two
problems have been addressed in OpenMC.

%%%%%%%%%%%%%%%%%%%%%%%%%%%%%%%%%%%%%%%%%%%%%%%%%%%%%%%%%%%%%%%%%%%%%%%%%%%%%%%%
\section{Methodology}

For fissile isotopes, OpenMC makes modifications to the heating reaction to
include all relevant components of fission energy release. These modifications
are made to the total heating reaction, MT=301. Breaking the total heating
KERMA into a fission and non-fission section, one can write
\begin{equation}
    k_i(E) = k_{i, nf}(E) + \left[E_{fr}(E) + E_{\gamma, p}\right]\sigma_{i, f}(E)
\end{equation}
OpenMC seeks to modify the total heating data to include energy from
$\upbeta$ particles and, conditionally, delayed photons. This conditional
inclusion depends on the simulation mode: neutron transport, or coupled
neutron-photon transport. The heating due to fission is removed using MT=318
data, and then re-built using the desired components of fission energy release
from MF=1,MT=458 data.

For the case of neutron-only transport, OpenMC instructs HEATR to produce
heating coefficients assuming that energy from photons, $E_{\gamma, p}$ and
$E_{\gamma, d}$, is deposited at the fission site. Let $N901$ represent the
total heating number returned from this HEATR run with $N918$ reflecting fission
heating computed from NJOY. $M901$ represent the following modification
\begin{equation}
    M901_{i}(E)\equiv N901_{i}(E) - N918_{i}(E)
      + \left[E_{i, fr} + E_{i, \beta} + E_{i, \gamma, p}
      + E_{i, \gamma, d}\right]\sigma_{i, f}(E).
\end{equation}
This modified heating data is stored as the MT=901 reaction and will be scored
if \texttt{heating-local} is included in the tally scores.

In a coupled neutron-photon transport simulation, OpenMC instructs HEATR to
assume that energy from photons is not deposited locally. However, the
definitions provided in the NJOY manual indicate that, regardless of this mode,
the prompt photon energy is still included in $k_{i, f}$, and therefore must be
manually removed. Let $N301$ represent the total heating number returned from
this HEATR run and $M301$ be
\begin{equation}
    M301_{i}(E)\equiv N301_{i}(E) - N318_{i}(E)
      + \left[E_{i, fr}(E) + E_{i, \beta}(E)\right]\sigma_{i, f}(E).
\end{equation}
This modified heating data is stored as the MT=301 reaction and will be scored
if \texttt{heating} is included in the tally scores.

OpenMC uses an HDF5 format for nuclear data that can be converted from ACE files
produced by NJOY. In the Python API, the \texttt{IncidentNeutron} class holds
incident neutron data and can be constructed starting with an ACE file by using
a \texttt{from\_ace} method. A \texttt{from\_njoy} method also exists and can
create neutron data starting from an ENDF file by running NJOY under the hood.
This method gives OpenMC more control over options in NJOY.

For energy deposition, the most accurate treatment is to use KERMA values that
account for the energy deposition from all reactions. There are two
desired modes of operation:
\begin{enumerate}
    \item In a neutron-only calculation, the KERMA values should be calculated
    assuming that photons deposit their energy locally. Without this assumption,
    some other method would needed to estimate how much energy is deposited from
    photons.
    \item In a coupled neutron-photon calculation, the KERMA values should be
    calculated assuming photons carry energy away from the reaction site, and
    photons should be allowed to deposit energy separately.
\end{enumerate}
When \texttt{IncidentNeutron.from\_njoy} is called, NJOY is run with a sequence
that includes two HEATR steps, one that assumes photons carry energy away and
one that assumes they do not. In principle, the KERMA values provided by NJOY
would be sufficient for calculating energy deposition. However, when fission
reactions are present, NJOY does not include energy from the delayed emission of
photons and electrons. This would be desired if it were to possible to calculate
the heating due to the decay of fission products directly using the ENDF decay
sublibrary files. However, this approach is complex and hasn't been widely
studied to date. Thus, normally it is assumed that the delayed energy is release
instantaneously, which is an appropriate assumption for reactors operating at
steady state.

Rather than using the NJOY provided KERMA values directly from MT=301, OpenMC
first makes modifications to them to account for delayed energy release in
fission. OpenMC will calculate

Processing of data through openmc.data and NJOY. MT=301 heating data and MT=901
heating-local.

Accounting for delayed photons through scaling photon production~\cite{trumbull2013mc}

Inconsistency with MF=458 data and photon production in ENDF/B-VII.1 (is this
mentioned in Serpent paper?)

Approach for tallying. Analog heating tally for photons, electrons, positrons
(discuss that KERMA was originally attempted, but TTB is a problem). KERMA-based
heating tally for neutrons

%%%%%%%%%%%%%%%%%%%%%%%%%%%%%%%%%%%%%%%%%%%%%%%%%%%%%%%%%%%%%%%%%%%%%%%%%%%%%%%%
\section{Results}


%%%%%%%%%%%%%%%%%%%%%%%%%%%%%%%%%%%%%%%%%%%%%%%%%%%%%%%%%%%%%%%%%%%%%%%%%%%%%%%%
\section{Conclusions}

It would be better to use known compositions along with decay sublibrary to
calculate energy deposition from delayed emission of neutrons, photons, and beta
particles

%%%%%%%%%%%%%%%%%%%%%%%%%%%%%%%%%%%%%%%%%%%%%%%%%%%%%%%%%%%%%%%%%%%%%%%%%%%%%%%%
\section{Acknowledgments}

%%%%%%%%%%%%%%%%%%%%%%%%%%%%%%%%%%%%%%%%%%%%%%%%%%%%%%%%%%%%%%%%%%%%%%%%%%%%%%%%
\bibliographystyle{ans}
\bibliography{references}

\end{document}

