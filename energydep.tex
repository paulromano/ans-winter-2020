\documentclass{anstrans}
\usepackage{graphicx}
\usepackage{booktabs}
\usepackage{microtype}
\usepackage{upgreek}
\usepackage[capitalise,nameinlink]{cleveref}

\newcommand{\efr}{E_\text{fr}}
\newcommand{\enp}{E_\text{n,p}}
\newcommand{\ened}{E_\text{n,d}}
\newcommand{\egp}{E_\text{$\gamma$,p}}
\newcommand{\egd}{E_\text{$\gamma$,d}}
\newcommand{\eb}{E_\upbeta}
\newcommand{\enu}{E_\nu}

%%%%%%%%%%%%%%%%%%%%%%%%%%%%%%%%%%%%%%%%%%%%%%%%%%%%%%%%%%%%%%%%%%%%%%%%%%%%%%%%
\title{Energy Deposition in the OpenMC Monte Carlo Particle Transport Code}
\author{Paul K.~Romano,$\!^{*}$ Andrew E. Johnson,$\!^{\dagger}$ Amanda L.~Lund,$\!^{*}$ and Jingang Liang$^{\ddag}$}

\institute{
$^{*}$Computational Science Division, Argonne National Laboratory, 9700 S Cass Ave., Lemont, IL 60439, promano@anl.gov
\and
$^{\dagger}$Georgia Institute of Technology, 770 State St NW, Atlanta, GA 30318, dasindrew@gatech.edu \and
$^{\ddag}$Institute of Nuclear and New Energy Technology, Tsinghua University, Beijing, China, jingang@tsinghua.edu.cn
}

\disclaimer{The submitted manuscript has been created by UChicago Argonne, LLC,
Operator of Argonne National Laboratory (``Argonne'').  Argonne, a U.S.
Department of Energy Office of Science laboratory, is operated under Contract
No. DE-AC02-06CH11357.  The U.S. Government retains for itself, and others
acting on its behalf, a paid-up nonexclusive, irrevocable worldwide license in
said article to reproduce, prepare derivative works, distribute copies to the
public, and perform publicly and display publicly, by or on behalf of the
Government. The Department of Energy will provide public access to these results
of federally sponsored research in accordance with the DOE Public Access Plan.
http://energy.gov/downloads/doe-public-access-plan.}


\begin{document}
%%%%%%%%%%%%%%%%%%%%%%%%%%%%%%%%%%%%%%%%%%%%%%%%%%%%%%%%%%%%%%%%%%%%%%%%%%%%%%%%
\section{Introduction}

Monte Code particle transport codes are often used to calculate the distribution
of energy deposition, otherwise known as the power distribution, in nuclear
reactors. Multiplying the fission reaction rate by a constant will give a rough
estimate of the energy deposition. However, it does not account for the
redistribution of energy due to the transport of neutral secondary particles
away from a reaction site. In the present work, we describe the methods used to
calculate energy deposition in OpenMC, a community-developed open source Monte
Carlo particle transport code~\cite{romano2015ane1}. With the recent inclusion
of photon transport in OpenMC~\cite{lund2018anl}, it is now possible to obtain
very accurate energy deposition distributions with OpenMC.

As particles traverse a problem, some portion of their energy is deposited at
collision sites. This energy is deposited when charged particles, including
electrons and recoil nuclei, undergo electromagnetic interactions with
surrounding electons and ions. The total heating rate can be written as,
\begin{equation}
    H(E) = \phi(E)\sum_i \rho_i \sum_r k_{i, r}(E),
\end{equation}
where $\phi(E)$ is the scalar flux at energy $E$, $\rho_i$ is the density of
nuclide $i$, and $k_{i, r}$ is the KERMA (Kinetic Energy Release in Materials)
coefficient for reaction $r$ of isotope $i$. The heating rate, $H(E)$, has units
of energy per time, typically eV/s. KERMA has units of energy $\times$
cross-section (e.g., eV-barn) and can therefore be used much like a reaction
cross section for the purpose of tallying energy deposition.

KERMA coefficients can be computed using a nuclear data processing code like
NJOY (specifically the HEATR module), which uses an energy-balance method to
calculate KERMA as
\begin{equation}
    \label{eq:kerma}
    k_r(E) = \left(E + Q_r - \bar{E}_{r, \text{n}}
    - \bar{E}_{r, \gamma}\right)\sigma_{r}(E),
\end{equation}
where $Q_r$ is the $Q$ value of reaction $r$, $\bar{E}_{r,n}$ is the average
energy carried away by secondary neutrons, $\bar{E}_{r,\gamma}$ is the average
energy carried away by secondary photons, and $\sigma_r(E)$ is the microscopic
cross section for reaction $r$. Without loss of generality, we have omitted the
$i$ subscript for simplicity, and it is to be understood that $k_r$ represents
the KERMA for a specific nuclide. The term in parentheses is known as the
\emph{heating number} and represents the energy put in to the reaction (the
incident neutron energy and the $Q$ value) and the energy transported away by
secondary neutral particles.

During a fission event, there are potentially many secondary particles, and all
must be considered. The total energy released in a fission event is typically
broken up into the following categories:
\begin{equation*}
\begin{split}
  \efr  &\equiv \text{energy of fission fragments} \\
  \enp  &\equiv \text{energy of prompt fission neutrons} \\
  \ened &\equiv \text{energy of delayed fission neutrons} \\
  \egp  &\equiv \text{energy of prompt fission photons} \\
  \egd  &\equiv \text{energy of delayed fission photons} \\
  \eb   &\equiv \text{energy of released $\upbeta$ particles} \\
  \enu  &\equiv \text{energy of neutrinos}
\end{split}
\end{equation*}
These components are defined in MF=1, MT=458 data in an ENDF evaluation. All
these quantities may depend upon incident neutron energy, but this dependence is
not shown to make the following demonstrations cleaner. NJOY assumes that the
$Q$ value for fission is equal to the prompt energy release minus the incident
neutron energy:
\begin{equation}
    \label{eq:njoy-fissq}
    Q_f = \efr + \enp + \egp - E
\end{equation}
which results in the following expression for the fission KERMA:
\begin{equation}
    k_f(E) = \left[\efr + \enp + \egp - \bar{E}_{f,n} - \bar{E}_{f,\gamma}\right]\sigma_f(E).
\end{equation}
If the secondary neutron and photon yields and energies are consistent with the
components of fission energy release in MT=458, we then have $\enp = \bar{E}_n$
and $\egp = \bar{E}_p$, which results in a fission KERMA of
\begin{equation}
    \label{eq:njoy-kerma}
    k_f(E) = \efr \sigma_f(E).
\end{equation}

There are several problems with the use of \cref{eq:njoy-kerma} to tally energy
deposition from fission in a Monte Carlo transport simulation. First and
foremost, it does not account for the energy deposited by photons and electrons
that are emitted during the decay of fission products. If it were to possible to
calculate the heating due to the decay of fission products directly using the
ENDF decay sublibrary files, such an approach would be preferrable. However,
this approach is complex and hasn't been widely studied to date. It is common
practice then to assume that the energy of delayed particles is released
instantaneously, which is an appropriate assumption for reactors operating at
steady state. In addition to ignoring the energy of delayed particles, when
KERMA coefficients based on \cref{eq:kerma} are used in a neutron-only
calculation, the energy from prompt photons is also ignored. In this paper, we
discuss how these two problems have been addressed in OpenMC.

%%%%%%%%%%%%%%%%%%%%%%%%%%%%%%%%%%%%%%%%%%%%%%%%%%%%%%%%%%%%%%%%%%%%%%%%%%%%%%%%
\section{Methodology}

In OpenMC, energy deposition arising from neutron interactions is calculated by
using KERMA coefficients, and energy deposition arising from photon interactions
is calculated using an analog estimator. We begin by discussing the KERMA
coefficients used for neutron interactions.

\subsection{Neutron Interactions}

As discussed above, the KERMA values calculated by NJOY are not sufficient for
accurately calculating energy deposition. Thus, the first step is to calculate
modified KERMA coefficients, which is handled through OpenMC's Python-based data
processing module~\cite{romano2017epjwoc}. Two sets of KERMA coefficients are
needed~\cite{trumbull2013mc} for each nuclide corresponding to the two desired
modes of operation:
\begin{enumerate}
    \item In a neutron-only calculation, KERMA coefficients should be calculated
    assuming that photons deposit their energy locally. Without this assumption,
    some other method would be needed to estimate how much energy is deposited
    from photons.
    \item In a coupled neutron--photon calculation, KERMA coefficients should be
    calculated assuming photons carry energy away from the reaction site, and
    photons should be allowed to deposit energy along their path.
\end{enumerate}
When HDF5 nuclear data files are being generated for OpenMC, NJOY is run with a
sequence that includes two HEATR steps, one that assumes photons carry energy
away and one that assumes they do not. OpenMC then stores the following modified
KERMA in the HDF5 file for the coupled neutron--photon case:
\begin{equation}
    \label{eq:k}
    k(E) = k_\text{NJOY}(E) - k_{f,\text{NJOY}}(E) + \left ( \efr +
        \eb \right ) \sigma_f(E)
\end{equation}
where $k_\text{NJOY}$ and $k_{f,\text{NJOY}}$ are the total (MT=301) and fission
(MT=318) KERMA coefficients obtained from the NJOY/HEATR calculation with photon
energy carried away. A modified KERMA for the neutron-only case is calculated as
\begin{equation}
    \label{eq:klocal}
    \begin{split}
    k_\text{local}(E) = \; &k_\text{local,NJOY}(E) - k_{f,\text{local,NJOY}}(E) \\
        &+ \left ( \efr + \egp + \egd + \eb \right )  \sigma_f(E)
    \end{split}
\end{equation}
where $k_\text{local,NJOY}$ and $k_{f,\text{local,NJOY}}$ are the total and
fission KERMA coefficients obtained from the NJOY/HEATR calculation assuming
photon energy is deposited locally. At runtime, OpenMC can be instructed to use
$k(E)$ for tallying energy deposition in a coupled neutron--photon calculation
and $k_\text{local}(E)$ for a neutron-only calculation. The primary benefit of
performing a coupled neutron--photon calculation is that it accurately captures
the redistribution of energy from the transport of photons. Additionally, in a
problem with vacuum boundary conditions, some of the photon energy may leak out
of the problem, which is not accounted for when using $k_\text{local}(E)$ and a
neutron-only calculation.

In a neutron-only calculation, using \cref{eq:klocal} will account for the
energy release of delayed photons and betas. However, in a coupled
neutron--photon calculation \cref{eq:k} still does not account for delayed
photon energy release because ENDF incident neutron sublibrary files do not
include the yield or energy spectrum of delayed photons (since these depend on
material composition). However, the average energy of delayed photons and betas
is given in the components of fission energy release in MF=1, MT=458. To account
for the energy corresponding to the release of delayed photons, it is assumed
that their energy spectrum is the same as that of prompt
photons~\cite{tuominen2019ane}. This can be accomplished by scaling the yield of
prompt photons from fission,
\begin{equation}
    y'(E) = \frac{\egp + \egd}{\egp} y(E)
\end{equation}
where $y(E)$ is the normal prompt photon yield at energy $E$ and $y'(E)$ is a
scaled yield that accounts for the energy release of delayed photons.

In a $k$-eigenvalue calculation, there is an imbalance between energy release
and deposition related to the biasing of the fission source by $1/k_\text{eff}$
that has been discussed at length by Griesheimer et
al.~\cite{griesheimer2020physor}. To address this imbalance, non-fission energy
deposition needs to be weighted by a normalization factor. Although not
rigorously correct, using $k_\text{eff}$ as the normalization factor provides an
approximate correction to the energy balance. When energy deposition is being
tallied in OpenMC during a $k$-eigenvalue calculation, the non-fission portion
of KERMA is weighted by the most recent estimate of $k_\text{eff}$:
\begin{equation}
    \tilde{k}(E) = \left ( k(E) - k_f(E) \right) k_\text{eff} + k_f(E)
\end{equation}
where $k_f(E)$ is calculated as $\left ( \efr + \eb \right ) \sigma_f(E)$ for a
coupled neutron--photon calculation or $\left ( \efr + \egp + \egd + \eb \right)
\sigma_f(E)$ in a neutron-only calculation. Additionally, photons born from
non-fission reactions are given a particle weight that is a factor of
$k_\text{eff}$ higher than normal.


\subsection{Photon and Electron Interactions}

Determining energy deposition from photon and electron reactions is simpler than
neutrons because the dominant physical processes do not result in nuclear
transmutation. In principle, KERMA coefficients for photon interactions can be
calculated in the same manner as KERMA coefficients for neutron interactions.
However, this becomes difficult once one considers the tightly-coupled nature of
photon--electron transport. Instead, the approach in OpenMC is to use the pre-
and post-collision energies to estimate the energy deposited. Namely, at each
collision, the energy deposited is calculated as
\begin{equation}
    w \left ( E - E' - \sum_i \mathcal{E}_i \right ),
\end{equation}
where $w$ is the weight of the particle, $E$ is the pre-collision energy, $E'$
is the post-collision energy (or zero if the particle does not survive the
collision), and $\mathcal{E}_i$ is the energy of the $i$th secondary particle.
This score applies whether the particle in question is a photon, electron, or
positron. Although OpenMC does not transport electrons or positrons, it does use
a thick-target bremsstrahlung treatment that results in the production of
secondary photons. This implies that energy due to electron/positron
interactions is deposited at the site where they were originally created, a
reasonable assumption given their short range in matter. That being said, we
note that the method outlined here should work equally well even if electrons
and positrons were explicitly transported.

%%%%%%%%%%%%%%%%%%%%%%%%%%%%%%%%%%%%%%%%%%%%%%%%%%%%%%%%%%%%%%%%%%%%%%%%%%%%%%%%
\section{Results}

To test the energy deposition treatment in OpenMC, a cross-code comparison
between OpenMC and Serpent 2 carried out on two single pressurized water reactor
(PWR) assembly models from the VERA core physics benchmark progression
problems~\cite{godfrey2014casl}, namely models 2b and 2g. These models were
chosen because results using Serpent 2 and MCNP6 have been reported
previously~\cite{tuominen2019ane}. For both models, the geometry is a single
17$\times$17 fuel assembly with 3.1\% enriched UO$_2$ fuel. The problem has
reflective boundary conditions in the $x$ and $y$ directions and extends to
positive/negative infinity in the $z$ direction. Model 2b contains 25 empty
guide tubes; in model 2g, 24 of the guide tubes are replaced with
silver-indium-cadmium (Ag-In-Cd) control rods, leaving only the central guide
tube.

A coupled neutron--photon simulation was run for each of the benchmark models.
The run strategy was to use XX inactive generations, XX active generations, and
XX particles per generation. For both Serpent 2 and OpenMC, ENDF/B-VIII.0 cross
sections were used. All Serpent 2 results are taken directly
from~\cite{tuominen2019ane} rather than being calculated independently.
\cref{tab:vera2b,tab:vera2g} show the percentage of energy deposited in each of
the materials in the VERA 2b and 2g models, respectively, as calculated by
Serpent 2 and OpenMC. We see that the energy deposition calculated by OpenMC
agrees with Serpent 2 within statistical uncertainty. Furthermore, OpenMC
calculates an average of 199.XX $\pm$ 0.XX MeV per fission; again, this agrees
with the value of 199.817 $\pm$ 0.004 MeV per fission reported for Serpent 2
in~\cite{tuominen2019ane}.
\begin{table}[htbp]
  \centering
  \caption{Energy deposition in VERA problem 2b.}
  \label{tab:vera2b}
  \begin{tabular}{lcc}
    \toprule
    & Serpent 2 (\%) & OpenMC (\%) \\
    \midrule
    Fuel & 96.8920 $\pm$ 0.0000 & --- \\
    Cladding & 0.9405 $\pm$ 0.0002 & --- \\
    Coolant & 2.1675 $\pm$ 0.0002 & --- \\
    \bottomrule
  \end{tabular}
\end{table}

\begin{table}[htbp]
  \centering
  \caption{Energy deposition in VERA problem 2g.}
  \label{tab:vera2g}
  \begin{tabular}{lcc}
    \toprule
    & Serpent 2 (\%) & OpenMC (\%) \\
    \midrule
    Fuel & 94.079 $\pm$ 0.001 & --- \\
    Cladding & 1.0083 $\pm$ 0.0002 & --- \\
    Coolant & 2.6738 $\pm$ 0.0003 & --- \\
    Poison & 2.0728 $\pm$ 0.0004 & --- \\
    \bottomrule
  \end{tabular}
\end{table}


%%%%%%%%%%%%%%%%%%%%%%%%%%%%%%%%%%%%%%%%%%%%%%%%%%%%%%%%%%%%%%%%%%%%%%%%%%%%%%%%
\section{Conclusions}

A new capability to accurately calculate energy deposition has been implemented
in OpenMC. The methodology relies on calculating two sets of KERMA coefficients,
one for coupled neutron--photon transport simulations and one for neutron-only
simulations. Energy deposited in photon and electron interactions is calculated
directly using pre- and post-collision information. Modifications to photon
yields at runtime account for delayed photon energy release and prevent an
imbalance between energy release and deposition in $k$-eigenvalue calculations.

Simulations of two PWR assembly benchmark problems were carried out. Comparison
of the results obtained using OpenMC compare favorably with previously reported
results using Serpent 2.

It would be better to use known compositions along with decay sublibrary to
calculate energy deposition from delayed emission of neutrons, photons, and beta
particles

%%%%%%%%%%%%%%%%%%%%%%%%%%%%%%%%%%%%%%%%%%%%%%%%%%%%%%%%%%%%%%%%%%%%%%%%%%%%%%%%
\section{Acknowledgments}

This research was supported by the Exascale Computing Project (17-SC-20-SC), a
collaborative effort of the U.S. Department of Energy Office of Science and the
National Nuclear Security Administration. The submitted manuscript has been
created by UChicago Argonne, LLC, operator of Argonne National Laboratory under
contract DE-AC02-06CH11357.

%%%%%%%%%%%%%%%%%%%%%%%%%%%%%%%%%%%%%%%%%%%%%%%%%%%%%%%%%%%%%%%%%%%%%%%%%%%%%%%%
\bibliographystyle{ans}
\bibliography{references}

\end{document}

