\documentclass[10pt,noamssymb,svgnames]{beamer}
\usetheme{metropolis}

\usepackage{amsmath}
\usepackage{amssymb}
\usepackage{booktabs}
\usepackage{ragged2e}
\usepackage{upgreek}
\usepackage{algorithm}
\usepackage{algorithmicx}
\usepackage{algpseudocode}
\usepackage[absolute,overlay]{textpos}
\usepackage{tikz}
\usetikzlibrary{arrows,positioning,calc,decorations.pathmorphing}

\tikzset{onslide/.code args={<#1>#2}{%
  \only<#1>{\pgfkeysalso{#2}} % \pgfkeysalso doesn't change the path
}}
\tikzset{temporal/.code args={<#1>#2#3#4}{%
  \temporal<#1>{\pgfkeysalso{#2}}{\pgfkeysalso{#3}}{\pgfkeysalso{#4}} % \pgfkeysalso doesn't change the path
}}

\usepackage{hyperref}
\hypersetup{colorlinks=true,urlcolor=cyan,linkcolor=}

% Python colors
\definecolor{CyanBlueAzure}{HTML}{4B8BBE} % light blue
\definecolor{LapisLazuli}{HTML}{306998} % dark blue
\colorlet{cba}{CyanBlueAzure}
\colorlet{cn}{CyanBlueAzure!70!white}

\newcommand{\good}[1]{{\color{ForestGreen}{#1}}}
\newcommand{\bad}[1]{{\color{red}{#1}}}

\newcommand{\efr}{Q_\text{fr}}
\newcommand{\enp}{Q_\text{n,p}}
\newcommand{\ened}{Q_\text{n,d}}
\newcommand{\egp}{Q_\text{$\gamma$,p}}
\newcommand{\egd}{Q_\text{$\gamma$,d}}
\newcommand{\eb}{Q_\upbeta}
\newcommand{\enu}{Q_\nu}

% BEAMER CONFIGURATION ---------------------------------------------------------
\newcommand<>{\blue}[1]{{\color#2{blue}#1}}
\newcommand<>{\green}[1]{{\color#2{ForestGreen}#1}}
\setbeamerfont{block title}{size=\normalsize}
\setbeamerfont{block body}{size=\scriptsize}
\setbeamertemplate{mini frames}{}

%\setbeamercolor{sectiontitle}{fg=black}
\setbeamercolor{part title}{fg=black}
\setbeamercolor{progress bar}{fg=CyanBlueAzure}
\setbeamercolor{progress bar in head/foot}{fg=LapisLazuli,bg=CyanBlueAzure}
\setbeamercolor{progress bar in section page}{fg=LapisLazuli,bg=CyanBlueAzure}
\setbeamercolor{frametitle}{bg=black!75!white}

% Configure progress bars
\metroset{progressbar=frametitle}
\makeatletter
\setlength{\metropolis@progressonsectionpage@linewidth}{1pt}
\setlength{\metropolis@progressinheadfoot@linewidth}{2pt}
\makeatother

% Blank footnote
\newcommand\blfootnote[1]{%
  \begingroup
  \renewcommand\thefootnote{}\footnote{#1}%
  \addtocounter{footnote}{-1}%
  \endgroup
}

\bibliographystyle{ans}
\setbeamertemplate{bibliography item}{\insertbiblabel}

\newcommand{\vect}[1]{\mathbf{#1}} % vectors and matrices

%%---------------------------------------------------------------------------%%
\title{Energy Deposition in the OpenMC Monte Carlo Particle Transport Code}

\author{Paul Romano,\!$^1$ Andrew Johnson,\!$^2$ Amanda Lund,\!$^1$ and Jingang Liang$^3$}
\institute{$^1$Argonne National Laboratory \\ $^2$Georgia Institute of Technology \\ $^3$Tsinghua University}
\date{November 19, 2020}
\titlegraphic{%
  \begin{picture}(0,0)
    \put(325,-200){\makebox(0,0)[rt]{\includegraphics[height=1.5cm]{ecp_logo_color.eps}}}
  \end{picture}}

%%---------------------------------------------------------------------------%%
\begin{document}

\maketitle

%%---------------------------------------------------------------------------%%
%\section*{Outline}

% TOC, optional
%\begin{frame}{Outline}
%  \tableofcontents
%\end{frame}

\section{Background}

%%---------------------------------------------------------------------------%%
\begin{frame}{Energy Deposition}
  \begin{itemize}
    \item Neutron/photon transport codes are often used to obtain distribution
    of energy deposition \vfill
    \item Very simple first-order estimates are possible (e.g., multiply fission
    reaction rate by constant value) but ignore effects of energy
    redistribution from neutral secondary particles \vfill
    \item \emph{This paper describes recently implemented methods in OpenMC for
    accurately calculating energy deposition in coupled neutron--photon
    simulations}
  \end{itemize}
\end{frame}

\begin{frame}{Heating rate}
  \begin{itemize}
    \item Energy deposition is calculated by tallying a \emph{heating rate}
    \begin{equation*}
      H(E) = \blue<2>{\phi(E)}\sum_i \blue<3>{\rho_i} \sum_r \blue<4>{k_{i, r}(E)}
    \end{equation*}
    \only<2>{where $\phi(E)$ is the flux}%
    \only<3>{where $\rho_i$ is the density of nuclide $i$}%
    \only<4>{where $k_{i,r}(E)$ is the \emph{kerma} coefficient for reaction $r$}
    \item NJOY calculates kerma coefficients using an energy balance method:
    \begin{equation*}
      k_r(E) = \left(E + Q_r - \bar{E}_{r, \text{n}}
      - \bar{E}_{r, \gamma}\right)\sigma_{r}(E)
    \end{equation*}
    \item In principle, we're done; NJOY/HEATR already gives us kerma
    coefficients and we can use them when scoring tallies
    \item However, story is a little more complicated for fission
  \end{itemize}
\end{frame}

\begin{frame}{How NJOY calculates fission kerma}
  \only<1-4>{
  \begin{equation*}
    k_f(E) = \left(\blue<4>{E +} \blue{Q_f} - \bar{E}_{f, \text{n}}
    - \bar{E}_{f, \gamma}\right)\sigma_{f}(E)
  \end{equation*}
  }
  \only<2>{
    \begin{equation*}
      \begin{split}
        \efr  &\equiv \text{energy of fission fragments} \\
        \enp  &\equiv \text{energy of prompt fission neutrons} \\
        \ened &\equiv \text{energy of delayed fission neutrons} \\
        \egp  &\equiv \text{energy of prompt fission photons} \\
        \egd  &\equiv \text{energy of delayed fission photons} \\
        \eb   &\equiv \text{energy of released $\upbeta$ particles} \\
        \enu  &\equiv \text{energy of neutrinos}
      \end{split}
      \end{equation*}
  }
  \only<3-4>{
    \begin{equation*}
      \blue{\only<4>{E + }Q_f} = \efr + \enp + \egp\only<3>{ - E}
    \end{equation*}
  }
  \only<5-6>{
    \begin{equation*}
      k_f(E) = \left(\efr + \blue<6>{\enp} + \green<6>{\egp} - \blue<6>{\bar{E}_{f,\text{n}}}
       - \green<6>{\bar{E}_{f,\gamma}}\right)\sigma_f(E)
    \end{equation*}
  }
  \only<7>{
    \begin{equation*}
      k_f(E) = \efr \sigma_f(E)
    \end{equation*}
  }
\end{frame}

\begin{frame}{Issues with kerma}
  \begin{itemize}
    \item Two problems with NJOY-calculated kermas
    \begin{enumerate}
      \item In a neutron-only calculation, we would still be assuming that photons
      carry away energy
      \item NJOY-calculated kerma coefficients do not account for delayed energy
      release from fission (neutrons, photons, betas)
    \end{enumerate} \vfill
    \item \emph{How does OpenMC handle this?}
  \end{itemize}
\end{frame}

\section{Methodology}

\begin{frame}{OpenMC Approach}
  \begin{itemize}
    \item The approach used in OpenMC follows closely that introduced by
    Griesheimer and Stedry~\cite{griesheimer2013mc}
    \item Similar approach used in Serpent 2~\cite{tuominen2019ane}
  \end{itemize}
\end{frame}

\begin{frame}{Kermas}
  Different values are needed depending on whether we are simulating only
  neutrons or both neutrons and photons:
  \begin{itemize}
    \item In a neutron-only calculation, assume photons deposit energy locally
    \item In a coupled neutron--photon calculation, assume photons carry energy
    away, and transport of photons should deposit energy
  \end{itemize}
\end{frame}

\begin{frame}{Neutron-only calculation}
  \begin{itemize}
    \item Run NJOY/HEATR with a flag indicating photon energy is deposited locally
    \item Only fission kerma needs to be modified; other reactions we take kerma
    directly as calculated from NJOY
    \begin{equation*}
      \begin{split}
      k_{f,\text{local}}(E) = \; &k_\text{local,NJOY}(E) - k_{f,\text{local,NJOY}}(E) \\
      &+ \left ( \efr + \egp + \egd + \eb \right ) \sigma_f(E)
      \end{split}
    \end{equation*}
    \item Preprocessing handled by \texttt{openmc.data} Python module
    \item Stored as a special reaction (MT=901) in OpenMC data libraries
  \end{itemize}
\end{frame}

\begin{frame}{Coupled neutron--photon calculation}
  \begin{itemize}
    \item In this case, photon energy is carried away; we just need to account
    for delayed beta energy in fission kerma
    \begin{equation*}
      k_f(E) = k_\text{NJOY}(E) - k_{f,\text{NJOY}}(E) + \left ( \efr +
        \eb \right ) \sigma_f(E)
    \end{equation*}
    \item Store this as total heating (MT=301) in data library \vfill
    \item Two issues remain:
    \begin{enumerate}
      \item Photon production data only includes \emph{prompt} photons
      \item Equation above only gives us neutron heating; still need to account
      for photon heating itself
    \end{enumerate}
  \end{itemize}
\end{frame}

\begin{frame}{Delayed photon energy}
  \begin{itemize}
    \item ENDF files don't include fission delayed photon yield/spectrum
    \item To ensure delayed photon energy is accounted for, we scale the yield
    of photons
    \begin{equation*}
      y'(E) = \frac{\egp + \egd}{\egp} y(E),
    \end{equation*}
    \item Preserves total energy, but spectrum of delayed photons is not
    technically correct
  \end{itemize}
\end{frame}

\begin{frame}{Photon heating}
  \begin{itemize}
    \item In principle, can calculate photon kermas just as we do for neutrons
    \item However, tight coupling of photon--electron transport makes this
    difficult (atomic relaxation, thick-target bremsstrahlung)
    \item However, quite easy to calculate energy deposited on a per-collision
    (analog) basis:
    \begin{equation*}
      w \left ( E - E' - \sum_i \mathcal{E}_i \right )
    \end{equation*}
    where $w$ is weight of photon and $\mathcal{E}_i$ is energy of secondaries
    \item Can attribute energy deposition to photons, electrons, and positrons
    separately
  \end{itemize}
\end{frame}

\begin{frame}{Release/deposition imbalance}
  \begin{itemize}
    \item In a $k$-eigenvalue calculation, an imbalance exists between energy
    release and deposition because of biasing of fission source by
    $1/k_\text{eff}$
    \item Discussed at length by Griesheimer et al.~\cite{griesheimer2020physor}
    \item Nonfission portion of kerma is weighted by estimate of $k_\text{eff}$
    \begin{equation*}
      \tilde{k}(E) = \left ( k(E) - k_f(E) \right) k_\text{eff} + k_f(E),
    \end{equation*}
    \item Photons born from nonfission reactions are given a weight
    $k_\text{eff}$ higher than normal
  \end{itemize}
\end{frame}

\section{Results}

\begin{frame}{Test problems}
  \begin{itemize}
    \item Energy deposition comparisons made between OpenMC and Serpent
    \item VERA core physics problems, models 2b and 2g (17$\times$17 fuel
    assemblies)
    \item Coupled neutron--photon simulations carried out using ENDF/B-VIII.0
    interaction data
    \item Serpent results are taken directly from~\cite{tuominen2019ane}
  \end{itemize}
\end{frame}

\begin{frame}{VERA problem 2b results}
  \centering
  \begin{tabular}{lcc}
    \toprule
    & Serpent 2 (\%) & OpenMC (\%) \\
    \midrule
    Fuel & 96.8920 $\pm$ 0.0000 & 96.8919 $\pm$ 0.0003 \\
    Cladding & 0.9405 $\pm$ 0.0002 & 0.9404 $\pm$ 0.0001 \\
    Coolant & 2.1675 $\pm$ 0.0002 & 2.1678 $\pm$ 0.0003 \\
    \bottomrule
  \end{tabular}
\end{frame}

\begin{frame}{VERA problem 2g results}
  \centering
  \begin{tabular}{lcc}
    \toprule
    & Serpent 2 (\%) & OpenMC (\%) \\
    \midrule
    Fuel & 94.079 $\pm$ 0.001 & 94.0723 $\pm$ 0.0008 \\
    Cladding & 1.0083 $\pm$ 0.0002 & 1.0101 $\pm$ 0.0002 \\
    Coolant & 2.6738 $\pm$ 0.0003 & 2.6763 $\pm$ 0.0004 \\
    Poison & 2.0728 $\pm$ 0.0004 & 2.0752 $\pm$ 0.0004 \\
    \bottomrule
  \end{tabular}
\end{frame}

\begin{frame}{Conclusions}
  \begin{itemize}
    \item Methods have been implemented in OpenMC for calculating energy
    deposition in coupled neutron--photon simulations
    \item Two sets of kermas are generated, one for neutron-only cases and the
    other for coupled neutron--photon calculations
    \item Deposition from neutron collisions relies on kermas, deposition for
    photon collisions is handled directly for each collision
    \item Results on two test problems match results from Serpent 2
  \end{itemize}
\end{frame}

\begin{frame}{Acknowledgments}
  Special thanks to:
  \begin{itemize}
    \item ECP for funding\footnote{This research was supported by the Exascale Computing Project
    (17-SC-20-SC), a collaborative effort of two U.S. Department of Energy
    organizations (Office of Science and the National Nuclear Security
    Administration) responsible for the planning and preparation of a capable
    exascale ecosystem, including software, applications, hardware, advanced
    system engineering, and early testbed platforms, in support of the nation’s
    exascale computing imperative.}
    \item My co-authors for their contributions
  \end{itemize}
\end{frame}

\begin{frame}{References}
  \scriptsize
  \bibliography{references}
\end{frame}

\end{document}
